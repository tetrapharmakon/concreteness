\section{Generalities on concreteness}
We recall the main definition we will work with (see \cite{Bor1,McL}):
\begin{definition}\label{concrecat} A category $\C$ is called \emph{concrete} if
it admits a faithful functor $U\colon \C \to \Sets$.
\end{definition} Concreteness can be regarded as a smallness request; in fact,
the following remark shows that many of the categories arising in mathematical
practice are concrete simply because they are not big enough, either because
they are small, or because they are accessible (the proof of each of the
following statements is easy).
\begin{remark}[almost everything is concrete]\label{all-is-conc} Every small
category is concrete. Every accessible category is concrete. If a category is
not concrete, none of its small subcategories can be dense. Concreteness is a
self-dual property, \ie $\C$ is concrete if and only if $\C^\opp$ is concrete.
If $\C$ is a concrete, and $J$ is a small category, then the functor category
$\C^J$ is concrete. If $\C$ is monadic over $\Set$, then it is concrete.
\end{remark} 
The paper \cite{Isbell1964} states a condition for the concreteness of a category $\C$, which relies on the notion of a \emph{resolvable relation} between the classes of spans and cospans in $\C$. This was linked by \cite{freyd1973concreteness} to a smallness request on the class of so-called \emph{generalized regular subobjects} of objects in $\C$. In more detail, \cite{freyd1973concreteness} proves the following statements:
\begin{itemize}
	\item the Isbell condition of \cite{Isbell1964} is equivalent, in a category with finite products, to the smallness of the class of \emph{generalized regular subobjects} of each object $A\in\C$, that we define below in \refbf{grsdef}.
	\item In a category with finite limits, the Isbell condition is equivalent to the smallness of the class of regular subobjects of each object $A\in\C$.
	\item The smallness of each class of generalized regular subobjects is necessary for concreteness.
\end{itemize}
\begin{definition}[Regular Generalized Subobject]\label{grsdef} Let $f\in
\K_{/A}$ an object of the slice category, and let $C(f,B)$ be the class of pairs
$u,v : A \to B$ such that $uf=vf$. %$ as morphisms $\text{src}(f) \to A \to B$.
Define an equivalence relation $\asymp$ on objects of $\K_{/A}$ as
\begin{center}
$f\asymp g$ iff $C(f,B)=C(g,B)$ for every $B\in\K$, 
\end{center}
and let $\sub(\K_{/A})$ be the quotient of $\K_{/A}$ under this
equivalence relation. This is called the class of \emph{generalized regular
subobjects} of $A\in\K$.
\begin{remark}
Briefly, the $\asymp$ relation identifies two maps with codomain $A$ if and only if they equalize the same pairs of arrows. In the category of sets and functions, any morphism $f$ satisfies $f\asymp m_f$, where $m_f$ is the monomorphism appearing in the epi-mono factorization of $f$. More generally, the same argument shows that $f\asymp m_f$ in every category endowed with a factorization system with regular monomorphisms as right class. A slightly more general argument shows that in a finitely complete category with cokernel pairs, a morphism $f \colon X\to A$ is $\asymp$-equivalent to the regular monomorphism $q$ appearing in the equalizer
\[\notag
\begin{kodi}
\obj{E & A & |(push)| A\cup_X A \\};
\mor E q:-> A u:{swap,shove=4pt},-> push;
\mor A v:{shove=-4pt},-> push;
\end{kodi}
% \xymatrix{
% 	E \ar[r]^-q& A\ar@<4pt>[r]^-u\ar@<-4pt>[r]_-v & A\cup_XA.\\
% }
\]
In \cite{freyd1973concreteness} it is stated that in a category with finite limits the generalized regular subobjects of $A$ coincide with the regular subobjects of $A$ for each object $A$.
\end{remark}
\end{definition}
\begin{proposition}[Freyd condition]\label{isbell}
If $\K$ is concrete then its class of generalized regular subobjects $\sub(\K_{/A})$ is a set for every $A\in\K$.
\end{proposition}
It is worthwhile to notice that there is a completely dual definition of \emph{generalized regular quotients} $\quot(\K_{A/})$: one similarly defines a relation that identifies two maps with \emph{domain} $A$ if and only if they \emph{coequalize} the same pairs of arrows. The size of equivalence classes of generalized regular quotients characterize concreteness as well:
\begin{proposition}[co-Freyd condition]\label{coisbell}
If $\K$ is concrete then its class of generalized regular quotients $\quot(\K_{A/})$ is a set for every $A\in\K$.
\end{proposition}

\begin{remark}
Recall that if $\K$ has finite products, the Freyd condition is equivalent to the Isbell condition and thus to concreteness of $\K$. Since all the categories in this paper have finite products there is no real interest in distinguishing the two conditions. Instead of choosing cumbersome notation as Freyd-Isbell condition or similar, we conflate the two conditions, referring to the result, for the sake of brevity, as \emph{the Isbell condition}.
\end{remark}
Several universal constructions of $\Cat$ restrict to constructions on model categories: given the purpose of this work, we are principally interested in those constructions that transport non-concreteness. These includes particularly simple examples: equivalent categories are either both concrete or both non-concrete (so that every category which is Quillen equivalent to a given non-homotopy\hyp{}concrete one is non-homotopy concrete as well), and if $\mathcal{L}\hookrightarrow \K$ is a subcategory and $\mathcal{L}$ is not concrete, so is~$\K$.

We will sometimes exploit such straightforward results to prove that a model category $\M$ is not homotopy\hyp{}concrete. We need only a functor that is \emph{homotopy faithful}, meaning that it induces inclusion \emph{between localizations}; more than often there is no control on which maps $\mathcal{L}(X,Y)\to \M(X,Y)$ become monomorphisms $\ho(\mathcal{L})(X,Y)\to \ho(\M)(X,Y)$, so we need to single out a special case when this happens.
\begin{definition}[piercing model subcategory]
Let $\M$ be a model category; a \emph{piercing model subcategory} is a full subcategory $\W \overset{U}\hookrightarrow \M$, which is reflective and coreflective, and having the model structure for which an arrow $\varphi \colon W \to W'$ is in $\wk,\cof,\fib$ if and only if $U\varphi$ is in $\wk,\cof,\fib$ as an arrow of $\M$.
\end{definition}
\begin{remark}
In the terminology of \cite{may2011more}, a piercing model subcategory is a reflective and coreflective subcategory such that the inclusion $U$ \emph{strongly creates} the model structure on $\W$.
\end{remark}
\begin{definition}
Let $\W \hookrightarrow \M$ be a piercing model subcategory; we say that $\W$ is \emph{homotopy replete} if given a zig zag of weak equivalences in $\M$ from an object of $\W$, 
\[
W \leftrightarrow  \dots \leftrightarrow M
\]
the arrows, as well as the object $M$, lie in $\W$.
\end{definition}
\begin{proposition}
A piercing model subcategory $\W \overset{U}\hookrightarrow \M$ induces a faithful functor $\ho(\W) \overset{\ho(U)}\hookrightarrow \ho(\M)$.
\end{proposition}
\begin{proof}
Since $\W$ is piercing in $\M$, there is a commutative square
\[
\begin{kodi}
\obj{|(A)| \W(\tilde V,\hat W)& [6.5em] |(B)| \M(U\tilde V,U\hat W) \\ |(D)| \ho(\W)(\tilde V,\hat W) & |(C)| \ho(\M)(U\tilde V,U\hat W)\\};
\mor A -> B ->> C;
\mor[swap] * ->> D {\overline{U}_{VW}}:dashed,-> *;
\end{kodi}
\]
where the horizonal arrows are the actions of the functors $U,\ho(U)$ on hom-sets. The first isomorphism theorem for sets now yields that $\overline{U}_{VW}$ is injective.
\end{proof}
\begin{example}\label{gpd-in-cat}
The inclusion $\Gpd_\text{folk}\hookrightarrow \Cat_\text{folk}$ turns $\Gpd_\text{folk}$ into a piercing model subcategory.
\end{example}
\begin{remark}\label{unconcrete-frombelow}
As a consequence of this result, if a piercing model subcategory $\W \hookrightarrow \M$ is not homotopy concrete, then neither is $\M$.
\end{remark}
