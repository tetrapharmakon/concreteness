\section{$\cate{Ho}$ is not concrete}
The group of remarks in \refbf{all-is-conc} suggests that ``every'' category arising in mathematical practice should be concrete. And yet, in his \cite{fconc} Peter Freyd was able to offer a nontrivial example of a non-concrete category, consisting of topological spaces and homotopy classes of continuous functions.

Freyd's proof is based on several technical lemmas and it is in fact the result of an extremely clever manipulation of basic constructions on topological spaces. We now provide a short but detailed survey of his original idea.
Our aim in this section is to refurbish the classical proof of the theorem contained in \cite{fconc} that, spelled out in modern terms, asserts the following:
\begin{theorem}[Homotopy is not concrete]\label{honoconc}
Let $\wk$ denote the class of \emph{homotopy equivalences} in the category $\Top$ of topological spaces. Then the Gabriel-Zisman localization \cite{GZ} $\cate{Ho}=\Top[\wk^{-1}]$ is not concrete in the sense of \adef\refbf{concrecat}.
\end{theorem}
\begin{remark}
Freyd's proof seems to leave us free to choose \emph{any} model category structure on $\Top$ that has homotopy equivalences as weak equivalences (and, in particular, \cite{fconc} makes no mention of the classes of cofibrations), and even though such a model structure seems to exist \cite{strom1972homotopy}, the details of this proof are subject of debate (there's an instructive discussion on the $n$Lab page \cite{nlabstrommodel}). Therefore, we decide to restrict our attention to the subcategory of $\Top$ whose objects are \emph{compactly generated spaces}, where a more modern technology is available. We still denote this subcategory as $\Top$.
\end{remark}
\subsection{$\cate{Ho}$ is not concrete: the proof}
Freyd's strategy can be summarized in the following two points:
\begin{itemize}
	\item As stated above \refbf{grsdef}, concreteness for a category $\A$ is equivalent to \emph{Isbell condition}. This necessary and sufficient condition was first proved in \cite{Isbell1964} (necessity) and \cite{fconc} (sufficiency).
	\item In the homotopy category of spaces $\cate{Ho}$ it is possible to find an object (in fact, many) admitting a proper class of generalized regular subobjects.
\end{itemize}
For the remainder of the proof, we fix:
\begin{enumerate}
	\item An integer $n\ge 1$;
	\item an arbitrary prime $p$.
\end{enumerate}
To build an object with a proper class of generalized regular subobjects we manipulate the cofibration sequence of a suitable Moore space. The main tool here is a technical lemma that generates a proper class of groups having arbitrarily large height (see \cite{fuchs2015abelian}). 
\begin{lemma}[black box lemma]\label{spastic}
There exists a sequence $B_\bullet = (B_\alpha)$ of $p$-torsion abelian groups, one for each ordinal number $\alpha\in\cate{Ord}$, satisfying the following conditions:
\begin{itemize}
	\item each $B_\alpha$ contains an element $x_\alpha$ such that $p x_\alpha = 0$;
	\item when $\alpha < \beta$ every homomorphism of groups $f_{\alpha\beta} \colon B_\alpha \to B_\beta$ such that $f(px)=p f(x)$ sends $x_\alpha$ to zero.
\end{itemize}
\end{lemma}
We adopt this statement without further explanation (hence the name \emph{black box}): the interested reader will find a proof, based on the theory of \emph{heights} of torsion abelian groups, in \cite{Freydconc}.
\begin{notat}
For each ordinal $\alpha$, let now $M_\alpha$ be the Moore space $M(B_\alpha,n)$ on the group $B_\alpha$  in grade $n$ that we found inside the black box of Lemma \refbf{spastic}. Let $t_\alpha\colon \mathbb{Z}/p\mathbb{Z} \to B_\alpha$ be a group morphism having $x_\alpha$ in its image, and $u_\alpha\colon M \to M_\alpha$ the induced map $M(t_\alpha,n)$ between Moore spaces. We denote by $M$ the Moore space for $\mathbb{Z}/p\mathbb{Z}$ in degree $n$.
\end{notat}
\begin{remark}
Notice that in the canonical cofiber sequence
\[
M \xto{u_\alpha} M_\alpha \to C_\alpha \to \Sigma M \to \Sigma M_\alpha\to\dots
\]
the space $C_\alpha$ is a Moore space for $\coker(t_\alpha)$, and $\Sigma M_\alpha$ is a Moore space for $B_\alpha$. This is a key point in the proof.
\end{remark}
Now we claim that 
\[
\{C_{\alpha} \to \Sigma M \mid \alpha \in \cate{Ord}\}
\] is a proper class of generalized regular subobjects for $\Sigma M$. In order to prove this claim we need the following
\begin{proposition}
\label{key}
For each pair of ordinals $\alpha < \beta$ the composition
\[
C_\beta \xto{v_\beta} \Sigma M \xto{\Sigma u_{\alpha}} \Sigma M_\alpha
\]
is not null-homotopic.
\end{proposition}
\begin{proof}
We argue by contradiction: assume that the composition is homotopic to a constant map. Since $\Sigma M_\beta \simeq \text{cone}(v_\beta)$, we get a map $\Sigma M_\beta \to \Sigma M_\alpha$ that makes the left triangle below commute.
\[\label{the-diag}
\begin{tikzcd}
\Sigma M \ar[r]\ar[dr]& \Sigma M_\beta \ar[dotted, d]\\
& \Sigma M_\alpha
\end{tikzcd}
\qquad\qquad
\begin{tikzcd}
\mathbb{Z}/p\mathbb{Z} \ar[r]\ar[dr]& B_\beta \ar[dotted, d]\\
& B_\alpha
\end{tikzcd}
\]
But then applying the functor $H_{n+1}(\firstblank, \mathbb{Z})$ to this commutative triangle we get a contradiction on the right diagram of abelian groups whose solid arrows contain $x_\beta$ in their images, and yet the dotted arrow is the zero map on $x_\beta$.
\end{proof}
\begin{remark}
This is one of the most important remarks in the section. Until now our proof lived in $\Top$, but drawing diagram (\refbf{the-diag}), and in particular the arrow $\Sigma M_\beta \to \Sigma M_\alpha$, we have to move in the localization $\cate{Ho}$, as this arrow only exists there: in fact, there might be no map whatsoever between these two objects filling the triangle above, but only a zig-zag of continuous maps
\[
\Sigma M_\beta \xot{\simeq} \bullet \to \bullet \xot{\simeq} \bullet \dots \to \Sigma M_\alpha
\]
\end{remark}
Finally, we can conclude the proof.

\begin{proposition}
All the arrows $v_\alpha \colon C_\alpha \to \Sigma M$ form \emph{distinct} generalized regular subobjects of $\Sigma M$, so that $\sub(\cate{Ho}_{/\Sigma M})$ contains a proper class.
\end{proposition}
\begin{proof}
Suppose $v_\alpha \asymp v_\beta$ for $\alpha < \beta$. Since in the following diagram
\[
\begin{kodi}
\obj{
|(Ca)| C_\beta & &[3em] \\[-2em]
&|(SM)| \Sigma M &|(SMb)| \Sigma M_\alpha\\[-2em]
|(Cb)| C_\beta & & \\
};
\mor SM {\Sigma u_\alpha}:-> SMb;
\mor Ca v_\beta:-> +;
\mor[swap] Cb v_\alpha:-> SM;
\end{kodi}
\]
the composition of $\Sigma u_\alpha \circ v_\alpha$ is null-homotopic (we assumed that $v_\alpha$ and $v_\beta$ equalize the same arrows, hence they both equalize the pair $(0,\Sigma u_\alpha)$), also the composition $\Sigma u_\alpha \circ v_\beta$ is null-homotopic. This contradicts lemma \refbf{key}.
\end{proof}
This concludes Freyd's original proof, and paves the way to a certain number of questions and generalization.