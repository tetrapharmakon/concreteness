\documentclass{amsart}

\usepackage[left=3cm,right=3cm]{geometry}

\usepackage{xcolor,graphicx}

\begin{document}
\section*{Detailed remarks}
Paper:
\begin{itemize}
\item \textcolor{orange}{Remark 2.4 - an example might help here, e.g. for when K is the category of sets.}
\item We modified Remark 2.4 to include a proof that $\asymp$-equivalence classes are regular subobjects in the category of sets, and more generally in any category with regular-mono factorization.
\item \textcolor{orange}{I think using the abbreviation "grs" for "generalized regular subobject" and "WCO" for "weak classifying object" makes the paper slightly harder to follow, and a little bit less professional-looking. I recommend avoiding these abbreviations. Also, in remarks 4.6 and 4.7, the "wco" are lowercase. Same with grqs and abbreviating Theorem to Thm. It's better not to abbreviate.}
\item All occurrences of wco and grs are now removed and the term is not abbreviated
\item \textcolor{orange}{First sentence after 2.5 should say "it is worthwhile to notice"}
\item This has been corrected.
\item \textcolor{orange}{Remark 3.2 - there is a missing citation after "seems to exist". Also, please clarify what you mean by "seem to be an object of debate". Also, I don't think "paraphernalia" is the right word here.}
\item This has been corrected, and slightly improved the clarity of the phrase.
\item \textcolor{orange}{Lemma 3.3 - I don't understand what purpose the footnote serves}
\item The name of the lemma has been changed and we explained the reason for this choice.
\item \textcolor{orange}{Remark 3.4 - I can't fill in the details here, but the remark is not necessary for the proof. Please add a few more details about why this cannot be a functor.}
\item As correctly pointed out there's no need for the remark in the proof. We decided to remove the remark.
\item \textcolor{orange}{Proposition 3.9 - the proof would be a little easier to follow if you actually wrote out which pair of arrows is being equalized. Right now it's buried in the definition of the word "nullhomotopic", but with one more sentence you could help the reader follow.}
\item It is now explicitly spelled out what we meant and which pair of arrows, when equalized, conclude the argument.
\item \textcolor{orange}{Definition 4.4 - should the domain of the functor $\omega$ be Ho(M) instead of M? Because in notation 4.5 it seems that only Ho(M) appears.}
\item This has been corrected.
\item \textcolor{orange}{Notation 4.5 - I suppose you could make this definition for right or left or both. Please quickly clarify which is meant in the notation, to avoid confusion.}
\item This has been corrected and we carefully denoted all occurrences of left and right wcos.
\item \textcolor{orange}{Definition 4.11 - is quasi-stability related to what Mark Hovey called a pretriangulated category in his book? Or, to any other notion from the theory of triangulated categories? At the very least, it would be good to point out that every stable model category is quasi-stable.}
\item This is now mentioned.
\item \textcolor{orange}{Examples 5.2, 5.3, and 5.5 - all assert that a given functor is a WCO, but you've only defined WCOs relative to some other functor. It would be good to say what the other functor is in each case, or to use Notation 4.5 and say "of type n" in each case. Also, a word about whether you mean "right WCO" or "left WCO" or both would be nice. Generally speaking, I think a few more details about these examples are in order, since this is probably the most important part of the paper.}
\item A better understanding of Example 5.5 forced us to consistenly modify the final part of the paper: old example 5.5 occupies now a separate section (section 6) where we collect all the preliminaries and the results needed to describe the example properly. Note in particular:
\begin{itemize}
	\item An initial paragraph discussing how this examples shows that our main theorem isn't stated in the highest possible generality, and why we decided to keep the statement in the old form;
	\item a less sketchy proof that the homotopy category of 1-types is not concrete, and the non-concreteness of Cat as a corollary.
\end{itemize}
\item \textcolor{orange}{Example 5.9 - please cite a reference for the existence of the local model structure. Probably Jardine proved this.}
\item This has been added.
\end{itemize}
References:
\begin{itemize}
\item \textcolor{orange}{In Bor94 and ML98, I don't understand the parenthetical ending the reference. Is that meant to be page numbers?}
\item \textcolor{orange}{In Toen, there is a mysterious [46]}
\item Both these errors (due to an error in the .bib file) have been corrected.
\end{itemize}
Generally speaking, other small typos have been corrected, and the wording of a few phrases has been improved. A final paragraph with acknowledgements has been added in the end of the paper.
An important modification involves the paragraphs between 2.2 and 2.3, and 2,7 of the current version of the paper: what we originally called Isbell condition is if fact only necessary. This does not affect our main result and the subsequent discussion as we only need the implication which holds true in whole generality. In all the cases of interest we do not need this distinction, as the presence of finite products entails the sufficiency of Isbell condition. We are available to any further correction or additional explanation requested by the referee 
\end{document}