\section{Introduction}
\begin{aquote}{\cite{fconc}}
[the homotopy category of spaces $\cate{Ho}$] has always been the best example
of an \emph{abstract} category -- though its objects are spaces, the points of
the spaces are irrelevant because the maps are not functions -- best, because of
all abstract categories it is the one most often lived in by real
mathematicians. \emph{It is satisfying to know that its abstract nature is
permanent, that there is no way of interpreting its objects as some sort of set
and its maps as functions.}
\end{aquote}
As final as it may sound, Freyd's result that ``homotopy is not concrete'', and
in particular the paragraph above, doesn't address the fundamental problem of
\emph{how often} and \emph{why} the homotopy category of a category $\C$ endowed
with a class $\W_\C \subseteq \hom(\C)$ of weak equivalences is not concrete.

One of the strongest motivations in writing the present paper has been to fill
this apparent gap in the literature, clarifying which assumptions on a (model or
relative) category $(\M, \wk)$ give the homotopy category $\ho(\M) =
\M[\wk^{-1}]$ the same permanently abstract nature.

Our main claim here is that indeed Freyd's theorems generalizes quite easily,
and that several model categories can't have a concrete localization at weak
equivalences; moreover, in light of this result the reason why this happens is
now evident. In a somewhat sloppy parlance that calls a category $(\M,\wk)$
`homotopy\hyp{}concrete' when $\ho(\M)$ is concrete, our result can be
summarized as the statement that very few model categories are
homotopy\hyp{}concrete, and that this happens \emph{as a consequence} of the
fact that they encode an homotopy theory.

It is of course possible, at least in certain cases, to show that a given $\M$
is not homotopy\hyp{}concrete using ad-hoc arguments adapted to the particular
choice of the pair $(\M, \wk)$: Freyd's \cite{fconc} does this for the category
$\Cat$ with its `folk' model structure where $\wk$ is the class of equivalences
of categories. Apart from being quite involved, though, such an approach fails
to put a lot of similar results of the same conceptual ground.

The present paper adresses these questions. Our main theorem is
\refbf{ginnunga}:
\begin{theorem*}
Let $\M$ be a pointed model category; if there exist an index $n_0 \in \mathbb{N}_{\ge
1}$ and a `weak classifying object' for the functor $\pi_{n_0} \colon \M \to
\cate{Grp}$ (\adef\refbf{wcodef}), then $\M$ is not homotopy\hyp{}concrete.
\end{theorem*}
Freyd's argument is a completely formal construction relying on nifty but elementary algebraic construction in abelian group theory (Lemma \refbf{spastic}) and on the fact that the category
of spaces ``contains a trace'' of the category of abelian groups, via the
\emph{Moore functors} $M(\firstblank,n) \colon \cate{Ab}\to \Top$; it will be a
focal point of our generalization to be able to transport this to a more general
model category using similar properties of \emph{Eilenberg-Mac Lane objects},
and their interplay with the looping functor $\Omega$.

We are then able to apply the machinery of \athm\refbf{ginnunga} to several
explicit examples, thus showing that Freyd`s claim that ``homotopy is not
concrete'' remains true in the modern parlance of homotopical algebra. This
suggests how the permanent abstractness of homotopy theory is a reflection of
the permanent abstractness of \emph{homotopical algebra}.

More in detail, as a consequence of \athm\refbf{ginnunga} we offer
\begin{itemize}
	\item a proof that the homotopy category of chain complexes is not homotopy concrete;
	\item a proof that the homotopy category of $\Cat_\text{folk}$\footnote{As
already mentioned, this is a shorthand to refer to the category of small
categories with its `folk' model structure having weak equivalences the
equivalences of categories, and cofibrations the functors injective on objects.}
is not concrete, independent from (and surely more elegant than) the argument
presented in \cite[§4.1]{fconc}; this result follows as a corollary of the fact
that the category of groupoids is not homotopy concrete, and this, in turn,
follows from the fact that the category of \emph{1-types} is not homotopy
concrete (these two categories being Quillen equivalent).
	\item A proof that the stable category of spectra $\cate{Sp}$ is not homotopy
concrete. Freyd \cite{Freydconc} observes that the stable category obtained as
Spanier-Whitehead stabilization of \textsc{cw}-complexes of dimension $\ge 3$
can't be concrete; our \refbf{spectra} can be thought as a slight refinement
that makes no assumptions on dimension.
	\item A proof that the local model structure
\cite{jardine1987simplical,dugger2004hypercovers} on the category of simplicial
sheaves on a site is not homotopy concrete.
\end{itemize}
Of course, we do not see these results as unexpected, given the tight relation
between unstable and stable homotopy, between categories and (geometric
realization of) simplicial sets, and between algebraic topology and algebraic
geometry.

We feel this is an additional step towards a deeper understanding of the notion
of concreteness and foundational issues in homotopy theory, and an additional
hint, if needed, for how set theory and homotopy theory do (or do not)
interplay.