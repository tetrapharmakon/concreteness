\documentclass[12pt]{amsart}
\usepackage{latexsym, url}
\usepackage{amsthm}
\usepackage{amsmath}
\usepackage{amsfonts}
\usepackage[italian]{babel}
\usepackage[utf8]{inputenc}
\usepackage{graphicx}
\usepackage{amssymb}
\usepackage{tikz-cd}

\usetikzlibrary{calc,positioning}
\thispagestyle{plain}

\def\cate#1{\mathbf{#1}}

\newcommand{\Pf}{\noindent \mbox{\em \bf Proof. \hspace{2mm}} }
\newtheorem{theo}{Theorem}[section]
\newtheorem{lemma}[theo]{Lemma}
\newtheorem{assume}[theo]{Assumption}
\newtheorem{propo}[theo]{Proposition}
\newtheorem{defi}[theo]{Definition}
\newtheorem{coro}[theo]{Corollary}
\newtheorem{rem}[theo]{Remark}
\newtheorem{pb}[theo]{Problem}
\newtheorem{exam}[theo]{Example}
\newtheorem{exams}[theo]{Examples}
\newcommand\shift{\operatorname{shift}}
\newcommand\ck{\mathcal {K}}
 
\date{}
 
\begin{document}
\title{Negli occhi di Medusa}
\author[]
{JP Freyd}
\thanks{} 
\address{}
 
\begin{abstract}
Faccio poco, concentro l'attenzione su Freyd.
\end{abstract} 
\keywords{}
\subjclass{}

\maketitle

\section{Il caso che sappiamo fare}

Allora, adesso facciamo il discorso per $\cate{Htop}$ e alla fine leviamo le ipotesi. Quello che faremo è rileggere la dimostrazione di Freyd tentando di sottolineare i nodi centrali della dimostrazione (i.e. dando i nomi alle rotelle che fanno girare l'argomento).

Consideriamo il diagramma che segue.
\begin{center}
\begin{tikzcd}
\arrow[loop left]{l}{\Sigma} \cate{hTop} \ar[r, shift left, "\pi"] & \ar[l,shift left,"i"]\cate{Grp}^\mathbb{N} \arrow[loop right]{r}{\text{sh}}
\end{tikzcd}
\end{center}

\begin{rem}
Supponiamo che $i$ sia una sezione di $\pi$:
\begin{enumerate}
\item $\pi \circ \Sigma = \shift  \circ  \ \pi$
\item Esiste una trasformazione naturale $\text{id} \Rightarrow \Sigma$
\item La trasformazione naturale \emph{serpeggia}, nel senso che esiste una mappa che fa commutare il seguente diagramma.
\begin{center}
\begin{tikzpicture}
\node (X) at (0,0) {$X$};
\node[right=1.5cm of X] (Y) {$Y$};
\node[below=1.5cm of X] (SX) {$\Sigma Y$};
\node[below=1.5cm of Y] (SY) {$\Sigma Y$};
\draw[->] (X) -- (Y); 
\draw[->] (Y) -- (SY);
\draw[->] (X) -- (SX); 
\draw[->] (SX) -- (SY);
\node (C) at ($(X)!.5!(SY)$) {$C$};
\draw[->] (Y) -- (C);
\draw[->] (C) -- (SX);
\end{tikzpicture}
\end{center}
($C=\text{cofib}(f)$ è in questo caso hocolim$f$ guardato come diagramma $\Delta^1 \to \cate{Top}$).
\end{enumerate}
\end{rem}
\newpage
Prendiamo una successione transfinita di gruppi $\{X_{\alpha}\}$ abeliani tali che esista una mappa non nulla $X_\alpha \to X_{\alpha+1}$ e nessuna composizione può essere nulla e concentriamoli su una fila, diciamo l'n esima. E buttiamo giù il seguente diagramma.
\begin{center}
\begin{tikzcd}
X_1 \ar[rr]\ar[dd]&& \Sigma X_1 \ar[rr]\ar[dd]&& \Sigma^2 X_1\ar[dd]\\
&C\ar[ur]&\\
X_2 \ar[ur]\ar[rr]\ar[dd]&& \Sigma X_2 \ar[rr]\ar[dd]&& \Sigma^2 X_2\ar[dd]\\
&C\ar[uuur]&\\
X_3 \ar[ur]\ar[rr]&& \Sigma X_3 \ar[rr]&& \Sigma^2 X_3\\
\end{tikzcd}
\end{center}
Dove stiamo confondendo volontariamente $iX_{\alpha}$ con $X_{\alpha}$.

Ora dico che le mappe $X_{\alpha} \to \Sigma X_1$ sono una successione transifinita di sottoggetti e ciao.

\section{Generalizzare}

Con la stessa dimostrazione si prova che

\begin{theo}
Se $\ck$ è una categoria che verifica le ipotesi $(i)-(iii)$ nel remark, allora $\ck$ non è concreta.
\end{theo}

\hrulefill

Mi chiedo, a questo punto, se sia vero quanto segue
\begin{theo}
Sia $\ck$ una categoria modello che verifica le $(i)-(iii)$ del remark; in particolare 
\begin{enumerate}
\item $\Sigma : \ck \to \ck$ è il funtore $X \mapsto \mathrm{hocolim}\!
\left[\hskip-1em
\vcenter{%
\begin{tikzcd}
X \ar[r]\ar[d]& * \\ * & 
\end{tikzcd}%
}%
\hskip-1em\right]$
\item $C=C(f)$ è $\mathrm{hocolim}(f)$.
\end{enumerate}
Allora $\cate{h}\ck$ non è concreta.
\end{theo}




\end{document}